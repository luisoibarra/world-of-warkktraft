%===================================================================================
% FESTIVAL DE LA CLASE - MATCOM, UH
%===================================================================================
% Esta plantilla ha sido diseñada para ser usada en los artículos del Festival de 
% la Clase de MatCom.
%
% Por favor, siga las instrucciones de esta plantilla y rellene en las secciones
% correspondientes.
%
% NOTA: Necesitará el archivo 'fc_jcematcom.sty' en la misma carpeta donde esté este
%       archivo para poder utilizar esta plantila.
%===================================================================================



%===================================================================================
% PREÁMBULO
%-----------------------------------------------------------------------------------
\documentclass[a4paper,10pt,twocolumn]{article}

%===================================================================================
% Paquetes
%-----------------------------------------------------------------------------------
\usepackage{amsmath}
\usepackage{amsfonts}
\usepackage{amssymb}
\usepackage{amsthm}
\usepackage{fc_jcematcom}
\usepackage[utf8]{inputenc}
\usepackage{listings}
\usepackage[pdftex]{hyperref}
\usepackage{caption}
\usepackage{subcaption}
%-----------------------------------------------------------------------------------
% Configuración
%-----------------------------------------------------------------------------------
\hypersetup{colorlinks,%
	    citecolor=black,%
	    filecolor=black,%
	    linkcolor=black,%
	    urlcolor=blue}
%-----------------------------------------------------------------------------------
% Teoremas y definiciones
%-----------------------------------------------------------------------------------
\theoremstyle{theorem}
\newtheorem{thm}{Teorema}[section]
\theoremstyle{definition}
\newtheorem{cor}[thm]{Corolario}
\newtheorem{lem}[thm]{Lema}
\newtheorem{prop}[thm]{Proposición}
\newtheorem{defn}[thm]{Definición}
\newtheorem{ejem}[thm]{Ejemplo}
\newtheorem{ejer}[thm]{Ejercicio}
\theoremstyle{remark}
\newtheorem{obs}[thm]{Observación}
%===================================================================================



%===================================================================================
% Presentacion
%-----------------------------------------------------------------------------------
% Título
%-----------------------------------------------------------------------------------
\title{Tema de la clase\\
\vspace{2ex}
\begin{large}
Asignatura\\
Lic. en (...)
\end{large}}

%-----------------------------------------------------------------------------------
% Autores
%-----------------------------------------------------------------------------------
\author{\\
\name Autor \email \href{mailto:a.uno@lab.matcom.uh.cu}{autor@lab.matcom.uh.cu}
	\\ \addr Grupo B612}

%-----------------------------------------------------------------------------------
% Tutores
%-----------------------------------------------------------------------------------
\tutors{\\
Dr. Tutor Uno, \emph{Centro} \\
Lic. Tutor Dos, \emph{Centro}}

%-----------------------------------------------------------------------------------
% Headings
%-----------------------------------------------------------------------------------
\jcematcomheading{\the\year}{1-\pageref{end}}{Nombre del autor}

%-----------------------------------------------------------------------------------
\ShortHeadings{Tema de la clase}{Nombre del autor}
%===================================================================================



%===================================================================================
% DOCUMENTO
%-----------------------------------------------------------------------------------
\begin{document}

%-----------------------------------------------------------------------------------
% NO BORRAR ESTA LINEA!
%-----------------------------------------------------------------------------------
\twocolumn[
%-----------------------------------------------------------------------------------

\maketitle

%===================================================================================
% Lenguaje
%-----------------------------------------------------------------------------------
\selectlanguage{spanish} % Para producir el documento en Español
\vspace{0.5cm}
]

%===================================================================================
% Objetivos
%-----------------------------------------------------------------------------------
\section{Objetivos}
%-----------------------------------------------------------------------------------
\begin{enumerate}
	\item Esta sección va dedicada a los objetivos de la clase, las metas para el encuentro y ciertas especificidades que considere de importancia resaltar durante el trancurso de la clase.
	\item Según la temática se puede hacer alusión a los medios de enseñanza utilizados.
\end{enumerate}


%===================================================================================
% Introducción
%-----------------------------------------------------------------------------------
\section{Introducción}\label{sec:intro}
%-----------------------------------------------------------------------------------
(Este segmento tiene una duración de $x$ minutos.)\\

\textbf{¿Cómo introducir mi clase?}
		\begin{itemize}
			\item Recursos para motivar la clase.
			\item Recuento por los antecedentes de los resultados o investigadores.
			\item La introducción puede tener cualquier formato que se desee, no es necesario que se acompañe por plecas.
		\end{itemize}


%===================================================================================
% Desarrollo
%-----------------------------------------------------------------------------------
\section{Teorizando un poco}\label{sec:teoria}
%-----------------------------------------------------------------------------------
(Este segmento tiene una duración de $y$ minutos.)\\

Tras la introducción se podrán construir las secciones que se estimen convenientes para el desarrollo de la clase.

\begin{thm}[Un Teorema interesante]
	De esta forma puede enunciarse un resultado importante.
\end{thm}  

\begin{cor}
	También es posible derivar resultados. Los entornos \{prop\}, \{obs\}, \{ejem\} y \{ejer\} permiten enunciar de forma similar las proposiciones, observaciones, ejemplos y ejercicios respectivamente. El uso de estos ambientes es opcional.
\end{cor}

\begin{ejem}[Un brillante algoritmo] 
Para producir código fuente, envuélvalo en una figura flotante y etiquételo correctamente. Por ejemplo, en la Fig. \ref{fig:code} se muestra un código conocido.

	% Configuración de Listings
	\lstset{keywordstyle=\color{blue}, basicstyle=\small}

	\begin{figure}[htb]%
	\begin{lstlisting}[language=c]%

    int main(int argc, char** argv)
    {
        // Imprimiendo "Hola Mundo".
        printf("Hello, World");
    }

	\end{lstlisting}
	\caption{Código fuente de ejemplo.\label{fig:code}}
	\end{figure}
\end{ejem}

La cantidad de secciones y subsecciones en lo adelante y sus respectivos nombres quedan a elección del autor.

%-----------------------------------------------------------------------------------
	\subsection{Organización del documento}\label{sub:results}
%-----------------------------------------------------------------------------------
Para producir cuerpos flotantes (figuras o tablas), asegúrese de numerar y etiquetar correctamente cada figura. Las referencias a las figuras deben estar correctamente etiquetadas. Por ejemplo, véase la Fig. \ref{fig:code} o la Fig. \ref{fig:ex} dejada a continuación.
	\begin{figure}[h!]%
	\begin{center}
		\begin{tabular}{|c|c|c|} \hline
		 			& Método 1 	& Método 2 	\\ \hline
		A 			&  			&  			\\ \hline
		B			& 			& 			\\ \hline
		C 			& 			&  			\\ \hline
		\end{tabular}
	\caption{Figura de ejemplo. Recuerde especificar el origen de los datos que se muestran.}\label{fig:ex}
	\end{center}
	\end{figure}
 	
Pueden insertarse de esta manera también gráficos y otros archivos.	

%===================================================================================
% Ejercicios resueltos
%-----------------------------------------------------------------------------------
\section{Ejercicio 1}\label{sec:ejer_1}
%-----------------------------------------------------------------------------------
(Este segmento tiene una duración de $z_1$ minutos.)\\

Compartir las soluciones de ciertos ejercicios así como observaciones interesantes sobre ellas que puedan servir de guía para el desarrollo de la clase.


%-----------------------------------------------------------------------------------
\section{Ejercicio 2}\label{sec:ejer_2}
%-----------------------------------------------------------------------------------
(Este segmento tiene una duración de $z_2$ minutos.)\\

Muchas veces la interacción con los estudiantes puede ser de importancia. De este modo métodos variados de solución pueden aportar al enriquecimiento de la clase.\\

\textbf{Receso} (5 minutos)


%===================================================================================
% Conclusiones
%-----------------------------------------------------------------------------------
\section{Conclusiones} \label{sec:conc}
%-----------------------------------------------------------------------------------

(Este segmento tiene una duración de cierta cantidad minutos.)\\
 
Se resumirán los resultados más destacados ejercitados en la actividad.

Se puede hacer mención de aplicaciones del método estudiado, posibles investigaciones o repercusiones en la cotidianidad.


%===================================================================================
% Ejercicios Propuestos
%-----------------------------------------------------------------------------------
\section{Estudio independiente} \label{independ}
%-----------------------------------------------------------------------------------

Orientar y comentar los ejercicios propuestos que se deseen.

\begin{ejer}
	De creerlo conveniente puede asignarse tareas para el estudio independiente, o actividades de carácter evaluativo. 
\end{ejer}

\begin{ejer}
	 La cantidad de activiades será a conveniencia aunque podría ser de ayuda su justificar de manera coherente el volumen de trabajo.
\end{ejer}

El esquema de clase es variable y queda sujeto a la voluntad del participante, solo debe ajustarse a los requisitos de la convocatoria oficial.


%===================================================================================
% Bibliografía
%-----------------------------------------------------------------------------------
\begin{thebibliography}{99}
%-----------------------------------------------------------------------------------
	\bibitem{knuth} Donald E. Knuth. \emph{The Art of Computer Programming}.
		Volume 1: Fundamental Algorithms (3rd~edition), 1997.
		Addison-Wesley Professional.

	\bibitem{wiki} Wikipedia. URL: \href{http://en.wikipedia.org}
	  {http://en.wikipedia.org}.
		Consultado en \today.
		
\end{thebibliography}
%-----------------------------------------------------------------------------------

\label{end}

\end{document}

%===================================================================================